\addcontentsline{toc}{part}{Introduction Générale}
\markboth{\textbf{INTRODUCTION G\'EN\'ERALE}}{}
\chapter*{Introduction générale}
Les systèmes informatisés conservent généralement des données détaillées des interactions utilisateur-système, plus précisément des interactions système-apprenant dans les systèmes éducatifs et les systèmes de tutorat intelligent (ITS) en particulier. Ces données détaillées qui sont généralement dans une grande base de données offre des opportunités pour étudier ces données récolter. Dans les systèmes éducatifs et les systèmes de tutorat intelligent (ITS), les données sont enregistrées dans une base de données à chaque transaction qui est une interaction entre l'apprenant (étudiant) et le système de tutorat. La transaction se définit comme chaque tentative correcte ou incorrect, ou encore quand l’apprenant demande d’aide ou conseil (généralement appelé « hint » en anglais) sur un « item » qui se définit comme un une question ou une tache que l’apprenant doit accomplir. Les opportunités qu’offre les données collecter à partir d’un ITS sont : étudier le comportement de l’apprenant face au system, extraction des modèles supervisés et non supervisé à partir des données, estimations des paramètres Bayesian Knowledge Tracing (BKT) et de capter plus de nuances dans le comportement humais et d’estimer la capacité de l’apprenant à réussir un item et aussi d’estimer la difficulté, la discrimination des items et le paramètre de pseudo-chance. En général plusieurs model peuvent être obtenu à partir des données des ITS : des « model-based approach » et « similarity-based approach » qui sont obtenu après avoir mapper les éléments (« items ») aux compétences (« skills ») ; des modèles Bayesian Knowledge Tracing (BKT) ; des modèles de la théorie des réponses aux items TRI 1PL, 2PL, 3PL (modèles à un, deux ou trois paramètres). \\

\section*{Problématique}
L'évaluation pédagogique concerne l'inférence sur les connaissances, les compétences et les réalisations des élèves parce que les données ne sont jamais aussi complètes et sans équivoque qu'elles garantissent la certitude. La théorie des réponses aux items (Item Response Theory IRT) intervient dans cette situation pour faire un ajustement bayésien des réponses aux items en améliorant potentiellement la notation des tests et d’éclairer une décision sur l'intérêt de collecter le jeu de données où IRT a été appliquer. \\
Aussi, dans les systèmes éducatifs, et en particulier dans les systèmes de tutorat intelligent (ITS), beaucoup d’aptitudes ont une forte relation causale dans laquelle une aptitude doit être présentée avant une autre (hiérarchie des compétences selon les pré-requis), ce qui incite à séquencer l’apprentissage afin que les apprenants maitrisent les compétences préalables avant de passer aux compétences qui en dépendent. Dans ce cas plusieurs méthodes peuvent être utilisé comme les méthodes basées sur la similarité et ceux sur un modèle après avoir mapper les éléments aux compétences (Item-to skill mappings aussi appelé Q-matrix). Les méthodes basées sur la similarité, pour réaliser ce mappage, s’appuient sur l’hypothèse que les apprenants auront tendance à avoir des performances similaires sur des éléments qui nécessitent la même compétence. Ces méthodes cherchent d’abord à calculer une mesure ou un degré de similarité pour chaque paire d’items. Le calcule de mesure de similarité des éléments est effectuer en utilisant certains coefficients de similarité (Pearson, Kappa, Yule, Jaccard, Sokal et Fisher), tout en prenant en compte en plus des informations correctes et incorrectes (obtenu par l’apprenant sur un item) d’autres caractéristiques comportementales telles que le temps de réponse, le comportement de l’apprenant, le nombre de tentative sur un item, le hasard, etc.

\section*{Objectif et le Travail réalisé }
Nous avons utilisé IRT dans l’implémentation pour un ajustement bayésien des réponses aux items avant de décider si oui ou non le jeu de données valle le coup. L’inférence bayésienne qui est une méthode d’apprentissage des valeurs des paramètres dans les modèles statistiques à partir de données, est utilise avec le modèle de Rasch pour faire l’ajustement des réponses obtenu par les apprenants. \\
Après la validation du jeu de données, une matrice de similarité entre les items a été créer selon le critère suivant : chaque item à un nombre de réponse correct avec aide (hints) et sans aide, et incorrect avec aide et sans aide. Ensuite un clustering hard et soft a été effectuer avec la matrice de similarité.

\newpage

\section*{Plan du document : }

Ce document est organisé en deux parties :

\begin{description}
    \item[\textbf{La première partie :}] La première partie : présente l’état de l’art sur les différents domaines entrant en jeu dans le cadre de ce mémoire. Elle est composée de trois chapitres à savoir educational data minig, modèle de l’apprenant et découverte des prérequis, l’approche items-to-skills mapping, l’inférence bayésienne et la théorie de la réponse aux items, et Clustering Hard et Soft. \\
    Dans le premier chapitre nous présentons une définition des terminologies du data mining, du data mining éducatif et de leurs techniques, en précisant l'intégration du data mining dans l'éducation. Une présentation des concepts du modèle de l'apprenant, et une brève description des méthodes utilisées pour la modélisation de l'apprenant. Et comment les pré-requis de l'apprenant sont découverts dans le chapitre 2. Une définition des terminologies de la connaissance, composante de la connaissance et les approches pour le mappage des éléments aux compétences dans le troisième chapitre. \\
    Dans le quatrième chapitre nous avons défini et présenté les concepts de l’inférence bayésienne, de la théorie des réponses aux items, de l’application de l’inférence bayésienne pour faire l’estimation des paramètres des modèles IRT. Et dans le cinquième chapitre, nous présentons le clustering hard et soft plus précisément le clustering hiérarchique, partitionnel, « fuzzy clustering » (clustering flou) et une brève introduction d’autre méthode de clustering.     
    
    \item[\textbf{La deuxième partie : }]présente les contributions essentielles apportées. Elle est composée d’un seul chapitre dans lequel nous présentons les étapes parcourus jusqu’à l’obtention des résultats.   
\end{description}

Enfin, nous clôturons notre modeste travail par une conclusion générale ainsi que les différentes perspectives à développer dans l’avenir pour toute éventuelle amélioration de ce travail.
