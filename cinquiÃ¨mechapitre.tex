\section{Deep Learning}
	\LARGE{you have not cause u ask not }
L'apprentissage profond est l'une des techniques d'apprentissage automatique les plus populaires
les chercheurs universitaires et industriels utilisent en raison de sa capacité à apprendre un processus de calcul en profondeur qui imite les comportements naturels du cerveau humain [29]. Deeplearning4J (DL4J) est une bibliothèque de réseaux neuronaux distribués sous licence Apache 2.0, open source, écrite en Java et Scala. DL4J est une idée originale d'Adam Gibson de SkyMind et est le seul réseau d'apprentissage en profondeur de qualité commerciale qui s'intègre à Hadoop et Spark qui orchestre plusieurs threads hôtes. DL4J est une infrastructure d'apprentissage en profondeur unique, car elle utilise Map-Reduce pour entraîner le réseau tout en s'appuyant sur d'autres bibliothèques pour effectuer de grandes opérations matricielles.

Le framework DL4J est livré avec une prise en charge GPU intégrée, qui est une fonctionnalité importante pour le processus de formation et prend en charge YARN, l'infrastructure de gestion d'applications distribuée de Hadoop. DL4J dispose d'un riche ensemble de supports d'architecture de réseau profond: RBM, DBN, réseaux de neurones convolutifs (CNN), réseaux de neurones récurrents (RNN), RNTN et réseau de mémoire à long court terme (LTSM). DL4J inclut également la prise en charge d'une bibliothèque de vectorisation appelée Canova.

DL4J, implémenté en Java, est intrinsèquement plus rapide que Python. Il est aussi rapide que Caffe pour les tâches de reconnaissance d'images non triviales utilisant plusieurs GPU. Ce cadre offre d'excellentes capacités pour la reconnaissance d'image, la détection de fraude et le traitement du langage naturel.
Pour cet exemple on va utiliser un jeu de données Kaggle appelé Run or Walk [1] (Court ou marche). Kaggle est une plateforme web de Google qui héberge des concours de Machine Learning et des datasets. Ce jeu de données contient 88 588 entrées de l’accéléromètre et le gyroscope d’un iPhone 5c, ainsi que la date et l’heure à laquelle elles ont été prises. Les données sont prises dans des intervalles de 10 secondes avec une fréquence de 5,4 échantillons / seconde. Pour chaque entrée, il est indiqué si la personne court ou marche, ainsi que dans quelle main (poignet) elle tient le portable. Voici un exemple des premières lignes:
\section{\'Etude Expérimentale}
\section{Conclusion}
Les IDS sont actuellement des produits mûrs et aboutis. Ils continuent d'évoluer pour répondre aux exigences technologiques du moment mais offrent d'ores et déjà un éventail de fonctionnalités capables de satisfaire les besoins de tous les types d'utilisateurs. Néanmoins comme tous les outils techniques ils ont des limites que seule une analyse humaine peut compenser. Un peu comme les Firewalls, voir aussi ici, les détecteurs d'intrusion deviennent chaque jour meilleurs grâce à l'expérience acquise avec le temps mais ils deviennent aussi de plus en plus sensibles aux erreurs de configuration et de paramétrage

