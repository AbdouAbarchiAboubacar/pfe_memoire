\selectlanguage{french}
\begin{abstract}
\thispagestyle{plain}
\setcounter{page}{3}
%\setcounter
%\parindent=0.2cm	
\begin{singlespace}
    Les modèles de la théorie des réponses au items et les méthodes d’exploration de données éducatives sont très utilisés pour analyser les performances des apprenants sur des questionnaires et la différence qui existe entre les questionnaires (items) comme le degré de difficulté. Ces méthodes sont largement utilisées pour optimiser la réussite et améliorer l’environnement (le système) dans lequel les utilisateurs apprennent. L’inférence bayésienne appliquer aux modèles IRT permet de capter les attributs du sujet (sa compétence) et les propriétés des items (sa difficulté, son pouvoir de discrimination, la probabilité de deviner la réponse correcte) pour pouvoir prédire la probabilité de réussite et ainsi comparer les prédictions des modèles et les scores collecter. La comparaison entre les scores prédit des apprenants et ceux observés et surtout les valeurs des paramètres permettent d’évaluer les échelles des mesures qui sont utilisés par le système, la qualité des données collecter et améliorer potentiellement la notation des scores des apprenants. Une fois les critères précédents valider, les données peuvent donc être utilisé avec les méthodes d’exploration de données éducatives qui permettent de transformer des données brutes de grande quantité en informations utiles pour découvrir des modèles et établir des tendances et des relations pour résoudre des problèmes et de prédire les tendances futures. Les méthodes EDM (Educational Data Mining) comme le clustering peuvent être utilisé pour regrouper les items en composante de connaissance en utilisant des approches basées sur des modèles et des approches basées sur la similarité entre items et ainsi séquencer l’apprentissage afin que les apprenants maitrisent les compétences préalables avant de passer aux compétences qui en dépendent. \\ \\
    Ce projet de mémoire propose une approche pour évaluer la qualité des données éducative avec des modèles IRT (modèle de Rasch, modèle logistique a deux et trois paramètres) et un regroupement des items en composante de connaissances avec des méthodes de clustering hiérarchique, partitionel et floue respectivement le clustering hiérarchique agglomératif, k-means clustering, c-means clustering, et le nombre de réponse correcte et incorrecte avec aide et sans aide comme critère pour la construction de la matrice de similarité. \\ \\
    \textbf{Mots clés : Exploration de données éducatives, modèle d'apprenant, ACP, méthodes de clustering, théorie de la réponse aux items, estimations MCMC.}
\end{singlespace}

\end{abstract}
%%%%%%%%%%%%%%%%%%%%%%%%%%%%%%%%%%%%%%%
%abstract 
\selectlanguage{english}
\begin{abstract}
\thispagestyle{plain}
\setcounter{page}{4}
\parindent=0.2cm
\begin{singlespace}
    Models of item response theory and educational data mining methods are widely used to analyze learner performance on questionnaires and the difference between questionnaires (items) such as degree of difficulty. These methods are widely used to optimize success and improve the environment (the system) in which users learn. Bayesian inference applied to IRT models makes it possible to capture the subject's attributes (his competence) and the properties of the items (his difficulty, his power of discrimination, the probability of guessing the correct answer) in order to be able to predict the probability of success and thus compare model predictions and scores collect. The comparison between the predicted scores of the learners and those observed and especially the values of the parameters make it possible to evaluate the scales of measures that are used by the system, the quality of the data to collect and potentially improve the scoring of the learners' scores. Once the above criteria are validated, the data can therefore be used with educational data mining methods that transform large amounts of raw data into useful information to discover patterns and establish trends and relationships to solve problems. and predict future trends. EDM (Educational Data Mining) methods such as clustering can be used to group items into knowledge components using model-based approaches and approaches based on similarity between items and thus sequence learning so that learners master the prerequisite skills before moving on to the skills that depend on them. \\ \\
    This thesis project proposes an approach to evaluate the quality of educational data with IRT models (Rasch model, two and three parameters logistic model) and a grouping of items into knowledge component with hierarchical, partitional and fuzzy clustering methods. respectively the hierarchical agglomerative clustering, k-means clustering, c-means clustering, and the number of correct and incorrect answers with help and without help as criteria for the construction of the similarity matrix. \\ \\
    \textbf{Keywords: Educational Data mining, Learner model, PCA, Clustering Methods, Items response theory, MCMC estimations.} 
\end{singlespace}

\end{abstract}